\input{preambuloSimple.tex}

%----------------------------------------------------------------------------------------
%	TÍTULO Y DATOS DEL ALUMNO
%----------------------------------------------------------------------------------------

\title{	
\normalfont \normalsize 
\textsc{\textbf{Curso 2017-2018} \\ Grado en Ingeniería Informática \\ Universidad de Granada} \\ [25pt] % Your university, school and/or department name(s)
\horrule{0.5pt} \\[0.4cm] % Thin top horizontal rule
\huge Práctica 1 \\ Nuevas Tecnologías de la Programación % The assignment title
\horrule{2pt} \\[0.5cm] % Thick bottom horizontal rule
}

\author{Carlos Manuel Sequí Sánchez} % Nombre y apellidos

\date{\normalsize\today} % Incluye la fecha actual

%----------------------------------------------------------------------------------------
% DOCUMENTO
%----------------------------------------------------------------------------------------

\begin{document}

\maketitle % Muestra el Título
\textbf{Entorno de desarrollo utilizado:} IDE Intellij Idea.\\
\textbf{Opinión personal sobre la práctica:} Pienso que ha consistido en una práctica con un nivel de dificultad adecuado para los conocimientos impartidos en clase. Además, creo que el paradigma de programación puesto en práctica es muy potente y novedoso, por ello, con vistas de futuro, pienso que es muy oportuno aprender este tipo de programación en un lenguaje tan extendido como Java, aunque al principio resulte un poco costoso el cambio a este nuevo estilo.



\end{document}
